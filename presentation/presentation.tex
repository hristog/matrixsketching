\documentclass[first=dgreen,second=purple,logo=redque]{aaltoslides}
%\documentclass{aaltoslides} % DEFAULT
%\documentclass[first=purple,second=lgreen,logo=redque,normaltitle,nofoot]{aaltoslides} % SOME OPTION EXAMPLES

\usepackage[latin9]{inputenc}
\usepackage[T1]{fontenc}
\usepackage{graphicx}
\usepackage{amssymb,amsmath}
\usepackage{url}
\usepackage{lastpage}
\usepackage{subfigure}
\usepackage{algpseudocode}
%\usepackage{xcolor}

\definecolor{gold}{rgb}{0.3, 0.4, .1}
\definecolor{fore}{RGB}{249,242,215}
\definecolor{lgray}{RGB}{25, 25, 25}
\definecolor{back}{RGB}{51,51,51}
\definecolor{title}{RGB}{255,0,90}
\definecolor{dgreen}{rgb}{0.,0.6,0.}
\definecolor{gold}{rgb}{1.,0.84,0.}
\definecolor{JungleGreen}{cmyk}{0.99,0,0.52,0}
\definecolor{gold}{cmyk}{0.85,0,0.33,0}
\definecolor{RawSienna}{cmyk}{0,0.72,1,0.45}
\definecolor{Magenta}{cmyk}{0,1,0,0}

\title{Simple and Deterministic Matrix Sketching}

\author[H. Georgiev and H. Shen]{Hristo Georgiev and Huibin Shen}
\institute[ICS]{Department of Information and Computer Science\\
Aalto University, School of Science and Technology}

\aaltofootertext{H. Georgiev and H. Shen}{T-61.6020}{\arabic{page}/\pageref{LastPage}\ }

%\date{}
\newcommand{\vectornorm}[1]{\left\|#1\right\|}
\newcommand{\IndState}{\State\hspace{\algorithmicindent}}

\everymath\expandafter{\the\everymath \color{blue}}

\begin{document}

%%%%%%%%%%%%%%%%%%%%%%%%%%%%%%%%%%%%%%%%%%%%%%%%%%%%%%%%%%%%%%%%%%%%%%%%%%%%%%%%%%%%%%%%%%%%%

\aaltotitleframe

%%%%%%%%%%%%%%%%%%%%%%%%%%%%%%%%%%%%%%%%%%%%%%%%%%%%%%%%%%%%%%%%%%%%%%%%%%%%%%%%%%%%%%%%%%%%%

\begin{frame}{Content}
\begin{itemize}
\item Background.
\item Related work.
\item Frequent directions.
\item Experiments and Results.
\item Conclusion.
\end{itemize}
\end{frame}

%%%%%%%%%%%%%%%%%%%%%%%%%%%%%%%%%%%%%%%%%%%%%%%%%%%%%%%%%%%%%%%%%%%%%%%%%%%%%%%%%%%%%%%%%%%%%


%\begin{frame}{Background}
%\begin{itemize}
%  \item Many applications are based on matrix, \emph{e.g.} search engine (document-term matrix), social networks (adjacency matrix).
%  \item In the ``Big data'' setting, matrix could be very large.
%  \item One way to handle massive data is to utilize map-reduce like programming model.
%\end{itemize}
%\end{frame}

%%%%%%%%%%%%%%%%%%%%%%%%%%%%%%%%%%%%%%%%%%%%%%%%%%%%%%%%%%%%%%%%%%%%%%%%%%%%%%%%%%%%%%%%%%%%%

%\begin{frame}{Background}
%\begin{itemize}
%  \item This paper: approximate the original matrix by a much smaller one while still preserving the correlation (if assume centering in the feature space). 
%  \item Formally, consider a large matrix $A \in \mathbb{R}^{n\times m}$ with $n$ rows and $m$ columns, a sketch matrix $B \in \mathbb{R}^{\ell \times m}$ containing only $\ell \ll n$ rows such that $A^TA \approx B^TB$. 

%This paper shows (Proof shown later):
%\begin{align}
%B^TB \prec A^TA \quad \text{and} \quad || A^TA-B^TB || \leq 2||A||_f^2/\ell \nonumber
%\end{align}
%\end{itemize}
%\end{frame}

%%%%%%%%%%%%%%%%%%%%%%%%%%%%%%%%%%%%%%%%%%%%%%%%%%%%%%%%%%%%%%%%%%%%%%%%%%%%%%%%%%%%%%%%%%%%%


%%%%%%%%%%%%%%%%%%%%%%%%%%%%%%%%%%%%%%%%%%%%%%%%%%%%%%%%%%%%%%%%%%%%%%%%%%%%%%%%%%%%%%%%%%%%%

%%%%%%%%%%%%%%%%%%%%%%%%%%%%%%%%%%%%%%%%%%%%%%%%%%%%%%%%%%%%%%%%%%%%%%%%%%%%%%%%%%%%%%%%%%%%%

\begin{frame}{What is a sketch?}
\begin{itemize}
  \item A sketch of a matrix $A$ is another matrix $B$ which is significantly
  smaller than $A$, but still approximates it well.
  \item A \textit{good} sketch matrix is one on which some computations can
  be performed, \textit{without} much loss of precision.
   \emph{(as opposed to performing them on the original matrix)}
\end{itemize}
\end{frame}

%%%%%%%%%%%%%%%%%%%%%%%%%%%%%%%%%%%%%%%%%%%%%%%%%%%%%%%%%%%%%%%%%%%%%%%%%%%%%%%%%%%%%%%%%%%%%

\begin{frame}[allowframebreaks=1]{What would one need a sketch?}
\begin{itemize}
  \item Modern large data sets are often viewed as matrices
   \emph{(which are often extremely large)}
  \item{Examples:}
  \begin{itemize}
    \item Textual data in the bag-of-words model \emph{(Where the rows
    correspond to documents)}
    \item Large-scale image analysis \emph{(Each row correspnds to one image,
    and contains either pixel values, or other derived feature values)}
  \end{itemize}
  \item Low rank approximations are used in PCA, LSI, $k$-means clustering
\end{itemize}

\framebreak

\begin{itemize}
  \item Typically, compute the SVD of some large matrix $A$
  \item Then approximate using the first $k$ singular vectors
  \begin{itemize}
     \item where $k \geq t$, i.e. we are interested only in unit vectors such
     that $\vectornorm{Ax} \geq t$.
  \end{itemize}
  \item However, the distributed nature of such matrices renders SVD infeasible
\end{itemize}

\end{frame}



%%%%%%%%%%%%%%%%%%%%%%%%%%%%%%%%%%%%%%%%%%%%%%%%%%%%%%%%%%%%%%%%%%%%%%%%%%%%%%%%%%%%%%%%%%%%%

\begin{frame}{Properties of matrix sketching methods}
Designed to be \textit{pass-efficient}
\begin{itemize}
\item i.e. data can be read only a \textit{constant} number of times.
\end{itemize}
The \textit{streaming model}: only one pass is permitted!
\begin{itemize}
\item Sketching becomes more challenging, since each row can be processed only
once. \emph{(further, storage is severely limited)}
\end{itemize}

\end{frame}

%%%%%%%%%%%%%%%%%%%%%%%%%%%%%%%%%%%%%%%%%%%%%%%%%%%%%%%%%%%%%%%%%%%%%%%%%%%%%%%%%%%%%%%%%%%%%

\begin{frame}[allowframebreaks=1]{Existing approaches}
There are three main approaches:
\begin{itemize}
\item \textit{random-projection}
\item \textit{hashing}
\item \textit{sampling}
\end{itemize}

Proposed fourth approach, \textit{Frequent-directions}.

\framebreak

Random-projection: encaptures two classes of methods
\begin{enumerate}[(i)]
  \item Generate a sparser version of the matrix.
   \begin{itemize}
   \item It can then be stored \textit{more efficiently}, and
   \item can be multiplied \textit{faster} by other matrices.
    \end{itemize}
  \item Randomly combine matrix rows.
\end{enumerate}

\framebreak

Hashing:
\begin{itemize}
  \item Simple and efficient \textit{subspace embeddings}
     that can be applied in $O(nnz(A))$ time, for any matrix $A$.
\end{itemize}

\framebreak

Sampling in the context of the \textit{Column Subset Selection Problem}:
\begin{itemize}
  \item find a small subset of matrix rows (or columns) that approximates
  the entire matrix
  \item solved using a simple streaming solution:
   \begin{itemize}
   \item sample rows from the input matrix with probability proportional to
   their squared $\ell_{2}$ norm
   \end{itemize}
\end{itemize}
\end{frame}

%%%%%%%%%%%%%%%%%%%%%%%%%%%%%%%%%%%%%%%%%%%%%%%%%%%%%%%%%%%%%%%%%%%%%%%%%%%%%%%%%%%%%%%%%%%%%

\begin{frame}[allowframebreaks=1]{Item frequency estimation}
\begin{itemize}
    \item Used to uncover \textcolor{gold}{frequent} iteams in an item stream 
	\item Referred to as \textcolor{gold}{\textit{Frequent-items}}
	\item \textbf{Goal:} use $O(\ell)$ space to produce estimates $g_{j}$, such
	that
	\begin{itemize}\item{$|f_{j} - g_{j} | \leq n/l$}, for all $j$
	\textcolor{red}{simultaneously}\end{itemize}

\item The algorithm:
  \begin{algorithmic}
    \State \textbf{Input}: \textcolor{dgreen}{items} $a_{1}$, $a_{2}$, \ldots,
    $a_{m}$
    \IndState \textcolor{dgreen}{item appearances} $A_{1}$, $A_{2}$, \ldots,
    $A_{n}$
    \\\State \textbf{Repeat} until there are \textcolor{red}{less than}
    $\ell$ \textcolor{gold}{unique} items left \{
     \IndState Simulate a process of
     \textcolor{red}{\textit{'deleting'}} of $\ell$ appearances of
     \IndState \textcolor{gold}{different} items from the stream
    \State \}
  \end{algorithmic}
\end{itemize}

%   \begin{itemize}
%      \item The \textcolor{red}{trimmed} stream can be stored
%      \textcolor{gold}{concisely} in $O(\ell)$ space
%   \end{itemize}

\framebreak

\textbf{Claim:}

\begin{itemize}
\item \textbf{If} item $a_{j}$ appears in the \textcolor{gold}{final} trimmed
stream $g_{j}$ times
\begin{itemize}
  \item \textbf{Then} $g_{j}$ is a \textcolor{red}{good approximation} for its
  true frequency $f_{j}$ (even in the case of $g_{j}$ = 0).
\end{itemize}         
\end{itemize}

\textbf{Proof:}
\begin{itemize}
   \item Each item-type is deleted \textcolor{gold}{at most} once per iteration:
   \begin{itemize} \item $f_{j} - g_{j} \leq t$, where $t$ is the
   number of deleted items.\end{itemize}
\end{itemize}

\framebreak

\textbf{Proof (cont.):}

\begin{itemize}
   \item \textcolor{gray}{Each item-type is deleted gold at most
   once per iteration:}
   \begin{itemize} \item \textcolor{gray}{$f_{j} - g_{j} \leq t$, where $t$ is
   the number of deleted items.}\end{itemize}
   \item $\ell$ items are deleted in every batch, and the total number of
   deleted items is upper-bounded by $n$, therefore:
   \begin{itemize} \item $t\ell \leq n$, or $t \leq n/\ell$.\end{itemize} 
\end{itemize}

\begin{itemize}
  \item If one sets $\ell > 1/\epsilon$,
  \begin{itemize} \item Then any item that appears more than
  $\epsilon n$ times in the stream \textcolor{gold}{must} appear in
  the final sketch.\end{itemize}
\end{itemize}
  
\end{frame}

\begin{frame}[allowframebreaks=1]{\textit{Frequent-items} in matrix sketching}

\begin{itemize}
  \item \textbf{Input:} matrix $A$, consisting of a stream of
  \textcolor{gold}{rows} $A_{i}$,
  	\begin{itemize}\item where $A_{i} \in \{ e_{1}, \ldots, e_{m} \}$, for
  	$e_{j}$ being the $j$th \textcolor{gold}{standard basis} vector
  	
  	\item $A_{j} = e_{j}$, if the $i$th row of the matrix
  	\item Frequency: $f_{j} = \vectornorm{Ae_{j}}^{2}$\\
  \end{itemize}
  \item \textbf{Output:} $g_{j} = \vectornorm{Be_{j}}^{2}$, which is a
  	\textcolor{red}{good approximation} to $f_{j}$
  \begin{itemize}\item Let $n = \vectornorm{A}_{f}^{2}$, then $|f_{j} -
  g_{j}| \leq n / \ell$, which is equivalent to
  \begin{itemize}\item $\vectornorm{Ae_{j}}^{2} -
  \vectornorm{Be_{j}}^{2} \leq \vectornorm{A}_{j}^{2} / \ell$\end{itemize}
  \end{itemize}
\end{itemize}

\end{frame}

%%%%%%%%%%%%%%%%%%%%%%%%%%%%%%%%%%%%%%%%%%%%%%%%%%%%%%%%%%%%%%%%%%%%%%%%%%%%%%%%%%%%%%%%%%%%%

\begin{frame}[allowframebreaks=1]{The \textit{Frequent-directions} algorithm}
\begin{itemize}
  \item It allows for the process to be inverted
  \begin{itemize}
    \item Prescribe the threshold $t$ in advance and find the space spanned by all
vectors $x$, such that $\vectornorm{Ax} \geq t$.
    \item In this setting, computing the SVD is not necessary anymore!
   \end{itemize}
\end{itemize}

\framebreak
Represent the frequency of a row (direction):
\begin{itemize}
  \item Assume rows of $A$ are indicate vectors of the items: \\
  \begin{center}
  \vspace{2 mm}
  $A = \begin{pmatrix}
       1 & 0 & 0 & 0\\[0.3em] 
       0 & 1 & 0 & 0\\[0.3em]
       0 & 0 & 1 & 0\\[0.3em]
       0 & 1 & 0 & 0
     \end{pmatrix}$
  \end{center}
  \item Frequency of second item $e_2 = (0,1,0,0)^T$:
  $ \vectornorm{Ae_2}^2 = \vectornorm{(0,1,0,1)^T} ^2= 0^2 + 1^2 + 0^2 + 1^2 = 2$. 
  \item Generalize the rows (directions) to $\{x : \vectornorm{x}=1\} $ and the frequency of a direction is $\vectornorm{Ax}^2$.
\end{itemize}

\framebreak
Connection to SVD of $A$:
\begin{itemize}
  \item $A = U\Sigma V^T \Leftrightarrow U^TA = \Sigma V^T \Leftrightarrow Au = \sigma v$.
  \item $\vectornorm{Au}^2 = \vectornorm{\sigma v}^2 = \sigma^2$.
\end{itemize}
  \vspace{2 mm}
Change $u$ to $x$: \\
  \vspace{2 mm}

The frequency of a direction is indicated by the square of corresponding sigular value $\sigma^2$.

\framebreak
The algorithm:
  \begin{algorithmic}
    \State \textbf{Input}: $\ell$, $A \in \mathbb{R}^{n\times m}$
    \State $B \leftarrow $ all zeros matrix $\in \mathbb{R}^{\ell\times m}$
    \For{$i =1,\ldots,n$} 
    \State Insert $i^{th}$ row of $A$ into zero valued row of $B$
    \If {$B$ has no zero valued rows}
      \State $[U,\Sigma,V] \leftarrow SVD(B)$
      \textcolor{gray}{\State $C \leftarrow \Sigma V^T$ // for proof}
      \State $\delta \leftarrow \sigma_{\ell/2}^2$
      \State $\breve{\Sigma} \leftarrow \sqrt{\max(\Sigma^2-I_\ell\delta,0)}$
      \State $B\leftarrow \breve{\Sigma}V^T$
    \EndIf
    \EndFor
  \end{algorithmic}
\end{frame}

\begin{frame}[allowframebreaks=1]{Properties of the sketch matrix $B$}
  \vspace{2 mm}
1. $A^TA \succeq B^TB \succeq 0$. 
\footnotesize
\begin{proof}
  \begin{flalign}
   & B^TB \succeq 0 \Leftrightarrow x^TB^TBx \geq 0 \Leftrightarrow \vectornorm{Bx}^2 \geq 0. && \nonumber \\
   & A^TA \succeq B^TB  \Leftrightarrow x^T(A^TA-B^TB)x \geq 0 \Leftrightarrow \vectornorm{Ax}^2 -\vectornorm{Bx}^2 \geq 0. \nonumber &&
  \end{flalign}  
  \begin{flalign}
    \vectornorm{Ax}^2 -\vectornorm{Bx}^2 &= \sum_{i=1}^n[\langle A_i, x \rangle^2 + ||B^{i-1}x||^2 - ||B^{i}x||^2] \nonumber &&\\
    &= \sum_{i=1}^n[||C^ix||^2 - ||B^{i}x||^2]= \sum_{i=1}^n[x^T(C^{i^T}C^i-B^{i^T}B^i)x] \nonumber &&\\
    &= \sum_{i=1}^n[x^T(V\Sigma^T\Sigma V^T-V\breve{\Sigma}^T\breve{\Sigma} V^T)x] 
    = \sum_{i=1}^n[x^TV(\Sigma^2 -\breve{\Sigma}^2)V^Tx] \geq 0. && \nonumber
  \end{flalign}
\end{proof}


\framebreak

\normalsize
  \vspace{2 mm}
2. $ ||A^TA - B^TB || \leq 2|| A||_f^2/\ell$. 
\footnotesize
\begin{proof}
  \begin{flalign}
    &||A^TA - B^TB || \leq \sum_{i=1}^n\delta_i \leq 2||A||_f^2/\ell. && \nonumber  
  \end{flalign}
  \begin{flalign}
    ||A^TA - B^TB || &= \sigma_{max}(A^TA - B^TB) = x^T(A^TA-B^TB)x = \vectornorm{Ax}^2 -\vectornorm{Bx}^2  \nonumber && \\
    &= \sum_{i=1}^n[||C^ix||^2 - ||B^{i}x||^2] \leq \sum_{i=1}^n[||C^{i^T}C^i - B^{i^T}B^i||] && \nonumber \\
    &= \sum_{i=1}^n[||V(\Sigma^2 -\breve{\Sigma}^2) V^T||] =  \sum_{i=1}^n \delta_i. \nonumber && 
  \end{flalign}
\end{proof}

\framebreak

\normalsize
  \vspace{2 mm}
2. $ ||A^TA - B^TB || \leq 2|| A||_f^2/\ell$. 
\footnotesize
\begin{proof}
%  \begin{flalign}
%    &||A^TA - B^TB || \leq \sum_{i=1}^n\delta_i \leq 2||A||_f^2/\ell. && \nonumber 
%  \end{flalign}
  \begin{flalign}
    ||B^n||_f^2 &= \sum_{i=1}^n[||B^{i-1}x||_f^2 - ||B^{i}||_f^2] = \sum_{i=1}^n[(||C^i||_f^2 - ||B^{i-1}x||_f^2) - (||C^i||_f^2 - ||B^{i}||_f^2)]. && \nonumber \\
     &= \sum_{i=1}^n ||A_i||_f^2 - tr(C^{i^T}C^i - B^{i^T}B^i) = ||A||_f^2 - \sum_{i=1}^ntr(V(\Sigma^{i^2} -\breve{\Sigma}^{i^2}) V^T) && \nonumber \\
     &= ||A||_f^2 - \sum_{i=1}^ntr(\Sigma^{i^2} -\breve{\Sigma}^{i^2}) \leq ||A||_f^2 - (\ell/2)\sum_{i=1}^n\delta_i \nonumber.
  \end{flalign}
  \begin{flalign}
    \sum_{i=1}^n\delta_i \leq 2(||A||_f^2 - ||B||_f^2)/\ell \leq 2||A||_f^2/\ell. && \nonumber
  \end{flalign}
\end{proof}
\normalsize


\framebreak
In summary:
\begin{itemize}
  \item $A^TA \succeq B^TB \succeq 0$.
  \vspace{2 mm}
  \item $ ||A^TA - B^TB || \leq 2|| A||_f^2/\ell$. 
  \vspace{2 mm}
  \item Let $A = [A_1;A_2]$ and $B_1$, $B_2$ is the sketches of $A_1$ and $A_2$. A sketch $C$ of $B=[B_1;B_2]$ can be shown that (proof omitted):\\
  \begin{align}
    ||A^TA - C^TC || \leq 2|| A||_f^2/\ell. \nonumber 
  \end{align}
\end{itemize}
\end{frame}

\begin{frame}{Experiments}

\end{frame}
%%%%%%%%%%%%%%%%%%%%%%%%%%%%%%%%%%%%%%%%%%%%%%%%%%%%%%%%%%%%%%%%%%%%%%%%%%%%%%%%%%%%%%%%%%%%%

%%%%%%%%%%%%%%%%%%%%%%%%%%%%%%%%%%%%%%%%%%%%%%%%%%%%%%%%%%%%%%%%%%%%%%%%%%%%%%%%%%%%%%


\end{document}
